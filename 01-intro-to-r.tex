\documentclass{article}\usepackage[]{graphicx}\usepackage[]{xcolor}
% maxwidth is the original width if it is less than linewidth
% otherwise use linewidth (to make sure the graphics do not exceed the margin)
\makeatletter
\def\maxwidth{ %
  \ifdim\Gin@nat@width>\linewidth
    \linewidth
  \else
    \Gin@nat@width
  \fi
}
\makeatother

\definecolor{fgcolor}{rgb}{0.345, 0.345, 0.345}
\newcommand{\hlnum}[1]{\textcolor[rgb]{0.686,0.059,0.569}{#1}}%
\newcommand{\hlstr}[1]{\textcolor[rgb]{0.192,0.494,0.8}{#1}}%
\newcommand{\hlcom}[1]{\textcolor[rgb]{0.678,0.584,0.686}{\textit{#1}}}%
\newcommand{\hlopt}[1]{\textcolor[rgb]{0,0,0}{#1}}%
\newcommand{\hlstd}[1]{\textcolor[rgb]{0.345,0.345,0.345}{#1}}%
\newcommand{\hlkwa}[1]{\textcolor[rgb]{0.161,0.373,0.58}{\textbf{#1}}}%
\newcommand{\hlkwb}[1]{\textcolor[rgb]{0.69,0.353,0.396}{#1}}%
\newcommand{\hlkwc}[1]{\textcolor[rgb]{0.333,0.667,0.333}{#1}}%
\newcommand{\hlkwd}[1]{\textcolor[rgb]{0.737,0.353,0.396}{\textbf{#1}}}%
\let\hlipl\hlkwb

\usepackage{framed}
\makeatletter
\newenvironment{kframe}{%
 \def\at@end@of@kframe{}%
 \ifinner\ifhmode%
  \def\at@end@of@kframe{\end{minipage}}%
  \begin{minipage}{\columnwidth}%
 \fi\fi%
 \def\FrameCommand##1{\hskip\@totalleftmargin \hskip-\fboxsep
 \colorbox{shadecolor}{##1}\hskip-\fboxsep
     % There is no \\@totalrightmargin, so:
     \hskip-\linewidth \hskip-\@totalleftmargin \hskip\columnwidth}%
 \MakeFramed {\advance\hsize-\width
   \@totalleftmargin\z@ \linewidth\hsize
   \@setminipage}}%
 {\par\unskip\endMakeFramed%
 \at@end@of@kframe}
\makeatother

\definecolor{shadecolor}{rgb}{.97, .97, .97}
\definecolor{messagecolor}{rgb}{0, 0, 0}
\definecolor{warningcolor}{rgb}{1, 0, 1}
\definecolor{errorcolor}{rgb}{1, 0, 0}
\newenvironment{knitrout}{}{} % an empty environment to be redefined in TeX

\usepackage{alltt}
\usepackage[sc]{mathpazo}
\renewcommand{\sfdefault}{lmss}
\renewcommand{\ttdefault}{lmtt}
\usepackage[T1]{fontenc}
\usepackage{geometry}
\geometry{verbose,tmargin=2.5cm,bmargin=2.5cm,lmargin=2.5cm,rmargin=2.5cm}
\setcounter{secnumdepth}{2}
\setcounter{tocdepth}{2}
\usepackage[unicode=true,pdfusetitle,
 bookmarks=true,bookmarksnumbered=true,bookmarksopen=true,bookmarksopenlevel=2,
 breaklinks=false,pdfborder={0 0 1},backref=false,colorlinks=false]
 {hyperref}
\hypersetup{
 pdfstartview={XYZ null null 1}}

\makeatletter
%%%%%%%%%%%%%%%%%%%%%%%%%%%%%% User specified LaTeX commands.
\renewcommand{\textfraction}{0.05}
\renewcommand{\topfraction}{0.8}
\renewcommand{\bottomfraction}{0.8}
\renewcommand{\floatpagefraction}{0.75}

\makeatother
\IfFileExists{upquote.sty}{\usepackage{upquote}}{}
\begin{document}



\title{\title{}}



\maketitle
The results below are generated from an R script.

\begin{knitrout}
\definecolor{shadecolor}{rgb}{0.969, 0.969, 0.969}\color{fgcolor}\begin{kframe}
\begin{alltt}
---
title: \hlstr{"Introduction to R"}
author: \hlstr{"Data Carpentry contributors"}
minutes: 45
---


```\{r, echo = FALSE, purl = FALSE, message = FALSE\}
\hlkwd{source}(\hlstr{"setup.R"})
```
\end{alltt}


{\ttfamily\noindent\bfseries\color{errorcolor}{\#\# Error: attempt to use zero-length variable name}}\end{kframe}
\end{knitrout}
\begin{knitrout}
\definecolor{shadecolor}{rgb}{0.969, 0.969, 0.969}\color{fgcolor}\begin{kframe}
\begin{alltt}

> \hlcom{### Learning Objectives}
>
> * Define the following terms as they relate to R: object, assign, call,
>   function, arguments, options.
> * Create objects and assign values to them  in R.
> * Learn how to _name_ objects.
> * Save a script file for later use.
> * Use comments to inform script.
> * Solve simple arithmetic operations in R.
> * Call functions and use arguments to change their default options.
> * Inspect the content of vectors and manipulate their content.
> * Subset and extract values from vectors.
> * Analyze vectors with missing data.
\end{alltt}


{\ttfamily\noindent\bfseries\color{errorcolor}{\#\# Error: <text>:2:1: unexpected '>'\\\#\# 1: \\\#\# 2: >\\\#\# \ \ \ \textasciicircum{}}}\end{kframe}
\end{knitrout}
\begin{knitrout}
\definecolor{shadecolor}{rgb}{0.969, 0.969, 0.969}\color{fgcolor}\begin{kframe}
\begin{alltt}

\hlcom{## Creating objects in R}

```\{r, echo = FALSE, purl = TRUE\}
\hlcom{### Creating objects in R}
```

You can get output from R simply by typing math in the console:

```\{r, purl = FALSE\}
3 + 5
12 / 7
```

However, to do useful and interesting things, we need to assign _values_ to
_objects_. To create an object, we need to give it a name followed by the
assignment operator `<-`, and the value we want to give it:

```\{r, purl = FALSE\}
weight_kg <- 55
```

`<-` is the assignment operator. It assigns values on the right to objects on
the left. So, after executing `x <- 3`, the value of `x` is `3`.   For historical reasons, you can also use `=`
for assignments, but not in every context. Because of the
[slight](https://blog.revolutionanalytics.com/2008/12/use-equals-or-arrow-for-assignment.html)
[differences](https://renkun.me/2014/01/28/difference-between-assignment-operators-in-r/)
in syntax, it is good practice to always use `<-` for assignments.

In RStudio, typing <kbd>Alt</kbd> + <kbd>-</kbd> (push <kbd>Alt</kbd> at the
same time as the <kbd>-</kbd> key) will write ` <- ` in a single keystroke in a PC, while typing <kbd>Option</kbd> + <kbd>-</kbd> (push <kbd>Option</kbd> at the
same time as the <kbd>-</kbd> key) does the same in a Mac.

Objects can be given almost any name such as `x`, `current_temperature`, or `subject_id`. Here are some further guidelines on naming objects:

* You want your object names to be explicit and not too long.
* They cannot start with a \hlkwd{number} (`2x` is not valid, but `x2` is).
* R is case sensitive, so for example, `weight_kg` is different from `Weight_kg`.
* There are some names that cannot be used because they are the names of fundamental functions in \hlkwd{R} (e.g., `if`, `else`, `for`, see
[here](https://stat.ethz.ch/R-manual/R-devel/library/base/html/Reserved.html)
for a complete list). In general, even if it\hlstr{'s allowed, it'}s best to not use
other function \hlkwd{names} (e.g., `c`, `T`, `mean`, `data`, `df`, `weights`). If in
doubt, check the help to see if the name is already in use.
* It\hlstr{'s best to avoid \hlkwd{dots} (`.`) within names. Many function names in R itself have them and dots also have a special \hlkwd{meaning} (methods) in R and other programming languages. To avoid confusion, don'}t include dots in names.
* It is recommended to use nouns for object names and verbs for function names.
* Be consistent in the styling of your code, such as where you put spaces, how you name objects, etc. Styles can include \hlstr{"lower_snake"}, \hlstr{"UPPER_SNAKE"}, \hlstr{"lowerCamelCase"}, \hlstr{"UpperCamelCase"}, etc. Using a consistent coding style makes your code clearer to read for your future self and your collaborators. In R, three popular style guides come from [Google](https://google.github.io/styleguide/Rguide.xml), [Jean
Fan](https://jef.works/R-style-guide/) and the
[tidyverse](https://style.tidyverse.org/). The tidyverse style is very comprehensive and may seem overwhelming at first. You can install the [**`lintr`**](https://github.com/jimhester/lintr) package to automatically check for issues in the styling of your code.

> \hlcom{### Objects vs. variables}
>
> What are known as `objects` in `R` are known as `variables` in many other
> programming languages. Depending on the context, `object` and `variable` can
> have drastically different meanings. However, in this lesson, the two words
> are used synonymously. For more information see:
> https://cran.r-project.org/doc/manuals/r-release/R-lang.html\hlcom{#Objects}


When assigning a value to an object, R does not print anything. You can force R to print the value by using parentheses or by typing the object name:

```\{r, purl = FALSE\}
weight_kg <- 55    \hlcom{# doesn't print anything}
(weight_kg <- 55)  \hlcom{# but putting parenthesis around the call prints the value of `weight_kg`}
weight_kg          \hlcom{# and so does typing the name of the object}
```

Now that R has `weight_kg` in memory, we can do arithmetic with it. For
instance, we may want to convert this weight into \hlkwd{pounds} (weight in pounds is 2.2 times the weight in kg):

```\{r, purl = FALSE\}
2.2 * weight_kg
```

We can also change an object's value by assigning it a new one:

```\{r, purl = FALSE\}
weight_kg <- 57.5
2.2 * weight_kg
```

This means that assigning a value to one object does not change the values of
other objects. For example, let\hlstr{'s store the animal'}s weight in pounds in a new
object, `weight_lb`:

```\{r, purl = FALSE\}
weight_lb <- 2.2 * weight_kg
```

and then change `weight_kg` to 100.

```\{r, purl = FALSE\}
weight_kg <- 100
```

What do you think is the current content of the object `weight_lb`? 126.5 or 220?

\hlcom{### Saving your code}

Up to now, your code has been in the console. This is useful for quick queries
but not so helpful if you want to revisit your work for any reason.
A script can be opened by pressing <kbd>Ctrl</kbd> + <kbd>Shift</kbd> +
<kbd>N</kbd>.
It is wise to save your script file immediately. To do this press
<kbd>Ctrl</kbd> + <kbd>S</kbd>. This will open a dialogue box where you
can decide where to save your script file, and what to name it.
The `.R` file extension is added automatically and ensures your file
will open with RStudio.

Don't forget to save your work periodically by pressing <kbd>Ctrl</kbd> +
<kbd>S</kbd>.


\hlcom{### Comments}

The comment character in R is `\hlcom{#`.  Anything to the right of a `#` in a script}
will be ignored by R. It is useful to leave notes and explanations in your
scripts.
For convenience, RStudio provides a keyboard shortcut to comment or uncomment a paragraph: after selecting the
lines you  want to comment, press at the same time on your keyboard
<kbd>Ctrl</kbd> + <kbd>Shift</kbd> + <kbd>C</kbd>. If you only want to comment
out one line, you can put the cursor at any location of that \hlkwd{line} (i.e. no need
to select the whole line), then press <kbd>Ctrl</kbd> + <kbd>Shift</kbd> +
<kbd>C</kbd>.

> \hlcom{### Challenge}
>
> What are the values after each statement in the following?
>
> ```\{r, purl = FALSE\}
> mass <- 47.5            \hlcom{# mass?}
> age  <- 122             \hlcom{# age?}
> mass <- mass * 2.0      \hlcom{# mass?}
> age  <- age - 20        \hlcom{# age?}
> mass_index <- mass/age  \hlcom{# mass_index?}
> ```

```\{r, echo = FALSE, purl = TRUE\}
\hlcom{### Challenge}
\hlcom{##}
\hlcom{## What are the values after each statement in the following?}
\hlcom{##}
\hlcom{## mass <- 47.5            # mass?}
\hlcom{## age  <- 122             # age?}
\hlcom{## mass <- mass * 2.0      # mass?}
\hlcom{## age  <- age - 20        # age?}
\hlcom{## mass_index <- mass/age  # mass_index?}
```

\hlcom{### Functions and their arguments}

Functions are \hlstr{"canned scripts"} that automate more complicated sets of commands
including operations assignments, etc. Many functions are predefined, or can be
made available by importing R *packages* (more on that later). A function
usually takes one or more inputs called *arguments*. Functions \hlkwd{often} (but not
always) return a *value*. A typical example would be the function `\hlkwd{sqrt}()`. The
\hlkwd{input} (the argument) must be a number, and the return \hlkwd{value} (in fact, the
output) is the square root of that number. Executing a \hlkwd{function} (\hlstr{'running it'})
is called *calling* the function. An example of a function call is:

```\{r, eval = FALSE, purl = FALSE\}
weight_kg <- \hlkwd{sqrt}(10)
```

Here, the value of 10 is given to the `\hlkwd{sqrt}()` function, the `\hlkwd{sqrt}()` function
calculates the square root, and returns the value which is then assigned to
the object `weight_kg`. This function takes one argument, other functions might take several. 

The return \hlstr{'value'} of a function need not be \hlkwd{numerical} (like that of `\hlkwd{sqrt}()`),
and it also does not need to be a single item: it can be a set of things, or
even a dataset. We'll see that when we read data files into R.

Arguments can be anything, not only numbers or filenames, but also other
objects. Exactly what each argument means differs per function, and must be
looked up in the \hlkwd{documentation} (see below). Some functions take arguments which
may either be specified by the user, or, if left out, take on a *default* value:
these are called *options*. Options are typically used to alter the way the
function operates, such as whether it ignores \hlstr{'bad values'}, or what symbol to
use in a plot.  However, if you want something specific, you can specify a value
of your choice which will be used instead of the default.

Let's try a function that can take multiple arguments: `\hlkwd{round}()`.

```\{r, results = \hlstr{'show'}, purl = FALSE\}
\hlkwd{round}(3.14159)
```

Here, we've called `\hlkwd{round}()` with just one argument, `3.14159`, and it has
returned the value `3`.  That's because the default is to round to the nearest
whole number. If we want more digits we can see how to do that by getting
information about the `round` function.  We can use `\hlkwd{args}(round)` to find what
arguments it takes, or look at the
help for this function using `?round`.

```\{r, results = \hlstr{'show'}, purl = FALSE\}
\hlkwd{args}(round)
```

```\{r, eval = FALSE, purl = FALSE\}
?round
```

We see that if we want a different number of digits, we can
type `digits = 2` or however many we want.

```\{r, results = \hlstr{'show'}, purl = FALSE\}
\hlkwd{round}(3.14159, digits = 2)
```

If you provide the arguments in the exact same order as they are defined you
don't have to name them:

```\{r, results = \hlstr{'show'}, purl = FALSE\}
\hlkwd{round}(3.14159, 2)
```

And if you do name the arguments, you can switch their order:

```\{r, results = \hlstr{'show'}, purl = FALSE\}
\hlkwd{round}(digits = 2, x = 3.14159)
```

It\hlstr{'s good practice to put the non-optional \hlkwd{arguments} (like the number you'}re
rounding) first in your function call, and to then specify the names of all optional
arguments.  If you don't, someone reading your code might have to look up the
definition of a function with unfamiliar arguments to understand what you're
doing.


\hlcom{## Vectors and data types}

```\{r, echo = FALSE, purl = TRUE\}
\hlcom{### Vectors and data types}
```

A vector is the most common and basic data type in R, and is pretty much
the workhorse of R. A vector is composed by a series of values, which can be
either numbers or characters. We can assign a series of values to a vector using
the `\hlkwd{c}()` function. For example we can create a vector of animal weights and assign
it to a new object `weight_g`:

```\{r, purl = FALSE\}
weight_g <- \hlkwd{c}(50, 60, 65, 82)
weight_g
```

A vector can also contain characters:

```\{r, purl = FALSE\}
animals <- \hlkwd{c}(\hlstr{"mouse"}, \hlstr{"rat"}, \hlstr{"dog"})
animals
```

The quotes around \hlstr{"mouse"}, \hlstr{"rat"}, etc. are essential here. Without the quotes R
will assume objects have been created called `mouse`, `rat` and `dog`. As these objects
don\hlstr{'t exist in R'}s memory, there will be an error message.

There are many functions that allow you to inspect the content of a
vector. `\hlkwd{length}()` tells you how many elements are in a particular vector:

```\{r, purl = FALSE\}
\hlkwd{length}(weight_g)
\hlkwd{length}(animals)
```

An important feature of a vector, is that all of the elements are the same type of data.
The function `\hlkwd{class}()` indicates what kind of object you are working with:

```\{r, purl = FALSE\}
\hlkwd{class}(weight_g)
\hlkwd{class}(animals)
```

The function `\hlkwd{str}()` provides an overview of the structure of an object and its
elements. It is a useful function when working with large and complex
objects:

```\{r, purl = FALSE\}
\hlkwd{str}(weight_g)
\hlkwd{str}(animals)
```

You can use the `\hlkwd{c}()` function to add other elements to your vector:
```\{r, purl = FALSE\}
weight_g <- \hlkwd{c}(weight_g, 90) \hlcom{# add to the end of the vector}
weight_g <- \hlkwd{c}(30, weight_g) \hlcom{# add to the beginning of the vector}
weight_g
```

In the first line, we take the original vector `weight_g`,
add the value `90` to the end of it, and save the result back into
`weight_g`. Then we add the value `30` to the beginning, again saving the result
back into `weight_g`.

We can do this over and over again to grow a vector, or assemble a dataset.
As we program, this may be useful to add results that we are collecting or
calculating.

An **atomic vector** is the simplest R **data type** and is a linear vector of a single type. Above, we saw
2 of the 6 main **atomic vector** types  that R
uses: `\hlstr{"character"}` and `\hlstr{"numeric"}` (or `\hlstr{"double"}`). These are the basic building blocks that
all R objects are built from. The other 4 **atomic vector** types are:

* `\hlstr{"logical"}` for `TRUE` and `FALSE` (the boolean data type)
* `\hlstr{"integer"}` for integer \hlkwd{numbers} (e.g., `2L`, the `L` indicates to R that it's an integer)
* `\hlstr{"complex"}` to represent complex numbers with real and imaginary \hlkwd{parts} (e.g.,
  `1 + 4i`) and that\hlstr{'s all we'}re going to say about them
* `\hlstr{"raw"}` for bitstreams that we won't discuss further

You can check the type of your vector using the `\hlkwd{typeof}()` function and inputting your vector as the argument.

Vectors are one of the many **data structures** that R uses. Other important
ones are \hlkwd{lists} (`list`), \hlkwd{matrices} (`matrix`), data \hlkwd{frames} (`data.frame`),
\hlkwd{factors} (`factor`) and \hlkwd{arrays} (`array`).


> \hlcom{### Challenge}
>
>
> * We’ve seen that atomic vectors can be of type character,
>   \hlkwd{numeric} (or double), integer, and logical. But what happens if we try to mix these types in
>   a single vector?
>
> ```\{text_answer, echo=FALSE, purl=FALSE\}
> R implicitly converts them to all be the same type
> ```
>
> * What will happen in each of these examples? (hint: use `\hlkwd{class}()`
>   to check the data type of your objects):
>
>     ```r
>     num_char <- \hlkwd{c}(1, 2, 3, \hlstr{"a"})
>     num_logical <- \hlkwd{c}(1, 2, 3, TRUE)
>     char_logical <- \hlkwd{c}(\hlstr{"a"}, \hlstr{"b"}, \hlstr{"c"}, TRUE)
>     tricky <- \hlkwd{c}(1, 2, 3, \hlstr{"4"})
>     ```
>
> * Why do you think it happens?
>
> ```\{text_answer, echo=FALSE, purl=FALSE\}
> Vectors can be of only one data type. R tries to
> \hlkwd{convert} (coerce) the content of this vector to find a "common
> denominator" that doesn't lose any information.
> ```
>
> * How many values in `combined_logical` are `\hlstr{"TRUE"}` (as a character) in the
>   following \hlkwd{example} (reusing the 2 `..._logical`s from above):
>
>     ```r
>     combined_logical <- \hlkwd{c}(num_logical, char_logical)
>     ```
>
> ```\{text_answer, echo=FALSE, purl=FALSE\}
> Only one. There is no memory of past data types, and the coercion happens the
> first time the vector is evaluated. Therefore, the `TRUE` in `num_logical`
> gets converted into a `1` before it gets converted into `\hlstr{"1"}` in
> `combined_logical`.
> ```
>
> * You've probably noticed that objects of different types get
>   converted into a single, shared type within a vector. In R, we
>   call converting objects from one class into another class
>   _coercion_. These conversions happen according to a hierarchy,
>   whereby some types get preferentially coerced into other
>   types. Can you draw a diagram that represents the hierarchy of how
>   these data types are coerced?
>
> ```\{text_answer, echo=FALSE, purl=FALSE\}
> logical &\hlcom{#8594; numeric &#8594; character &#8592; logical}
> ```

```\{r, echo = FALSE, eval = FALSE, purl = TRUE\}
\hlcom{## We’ve seen that atomic vectors can be of type character, numeric, integer, and}
\hlcom{## logical. But what happens if we try to mix these types in a single}
\hlcom{## vector?}

\hlcom{## What will happen in each of these examples? (hint: use `class()` to}
\hlcom{## check the data type of your object)}
num_char <- \hlkwd{c}(1, 2, 3, \hlstr{"a"})

num_logical <- \hlkwd{c}(1, 2, 3, TRUE)

char_logical <- \hlkwd{c}(\hlstr{"a"}, \hlstr{"b"}, \hlstr{"c"}, TRUE)

tricky <- \hlkwd{c}(1, 2, 3, \hlstr{"4"})

\hlcom{## Why do you think it happens?}


\hlcom{## How many values in `combined_logical` are `"TRUE"` (as a character) in the}
\hlcom{## following example:}
combined_logical <- \hlkwd{c}(num_logical, char_logical)

\hlcom{## You've probably noticed that objects of different types get}
\hlcom{## converted into a single, shared type within a vector. In R, we call}
\hlcom{## converting objects from one class into another class}
\hlcom{## _coercion_. These conversions happen according to a hierarchy,}
\hlcom{## whereby some types get preferentially coerced into other types. Can}
\hlcom{## you draw a diagram that represents the hierarchy of how these data}
\hlcom{## types are coerced?}
```

\hlcom{## Subsetting vectors}

If we want to extract one or several values from a vector, we must provide one
or several indices in square brackets. For instance:

```\{r, results = \hlstr{'show'}, purl = FALSE\}
animals <- \hlkwd{c}(\hlstr{"mouse"}, \hlstr{"rat"}, \hlstr{"dog"}, \hlstr{"cat"})
animals[2]
animals[\hlkwd{c}(3, 2)]
```

We can also repeat the indices to create an object with more elements than the
original one:

```\{r, results = \hlstr{'show'}, purl = FALSE\}
more_animals <- animals[\hlkwd{c}(1, 2, 3, 2, 1, 4)]
more_animals
```

R indices start at 1. Programming languages like Fortran, MATLAB, Julia, and R start
counting at 1, because that's what human beings typically do. Languages in the C
\hlkwd{family} (including C++, Java, Perl, and Python) count from 0 because that's
simpler for computers to do.

\hlcom{### Conditional subsetting}

Another common way of subsetting is by using a logical vector. `TRUE` will
select the element with the same index, while `FALSE` will not:

```\{r, results = \hlstr{'show'}, purl = FALSE\}
weight_g <- \hlkwd{c}(21, 34, 39, 54, 55)
weight_g[\hlkwd{c}(TRUE, FALSE, FALSE, TRUE, TRUE)]
```

Typically, these logical vectors are not typed by hand, but are the output of
other functions or logical tests. For instance, if you wanted to select only the
values above 50:

```\{r, results = \hlstr{'show'}, purl = FALSE\}
weight_g > 50    \hlcom{# will return logicals with TRUE for the indices that meet the condition}
\hlcom{## so we can use this to select only the values above 50}
weight_g[weight_g > 50]
```

You can combine multiple tests using `&` (both conditions are true, AND) or `|`
(at least one of the conditions is true, OR):

```\{r, results = \hlstr{'show'}, purl = FALSE\}
weight_g[weight_g > 30 & weight_g < 50]
weight_g[weight_g <= 30 | weight_g == 55]
weight_g[weight_g >= 30 & weight_g == 21]
```

Here, `>` for \hlstr{"greater than"}, `<` stands for \hlstr{"less than"}, `<=` for "less than
or equal to\hlstr{", and `==` for "}equal to". The double equal sign `==` is a test for
numerical equality between the left and right hand sides, and should not be
confused with the single `=` sign, which performs variable \hlkwd{assignment} (similar
to `<-`).

A common task is to search for certain strings in a vector.  One could use the
\hlstr{"or"} operator `|` to test for equality to multiple values, but this can quickly
become tedious. The function `%in%` allows you to test if any of the elements of
a search vector are found:

```\{r, results = \hlstr{'show'}, purl = FALSE\}
animals <- \hlkwd{c}(\hlstr{"mouse"}, \hlstr{"rat"}, \hlstr{"dog"}, \hlstr{"cat"}, \hlstr{"cat"})

\hlcom{# return both rat and cat}
animals[animals == \hlstr{"cat"} | animals == \hlstr{"rat"}]

\hlcom{# return a logical vector that is TRUE for the elements within animals}
\hlcom{# that are found in the character vector and FALSE for those that are not}
animals %in% \hlkwd{c}(\hlstr{"rat"}, \hlstr{"cat"}, \hlstr{"dog"}, \hlstr{"duck"}, \hlstr{"goat"}, \hlstr{"bird"}, \hlstr{"fish"})

\hlcom{# use the logical vector created by %in% to return elements from animals}
\hlcom{# that are found in the character vector}
animals[animals %in% \hlkwd{c}(\hlstr{"rat"}, \hlstr{"cat"}, \hlstr{"dog"}, \hlstr{"duck"}, \hlstr{"goat"}, \hlstr{"bird"}, \hlstr{"fish"})]
```

> \hlcom{### \hlkwd{Challenge} (optional)\{.challenge\}}
>
> * Can you figure out why `\hlstr{"four"} > \hlstr{"five"}` returns `TRUE`?
>
> ```\{text_answer, echo=FALSE, purl=FALSE\}
> When using \hlstr{">"} or \hlstr{"<"} on strings, R compares their alphabetical order.
> Here \hlstr{"four"} comes after \hlstr{"five"}, and therefore is \hlstr{"greater than"} it.
> ```

```\{r, echo = FALSE, purl = TRUE\}
\hlcom{### Challenge (optional)}
\hlcom{##}
\hlcom{## * Can you figure out why `"four" > "five"` returns `TRUE`?}
```


\hlcom{## Missing data}

As R was designed to analyze datasets, it includes the concept of missing data
(which is uncommon in other programming languages). Missing data are represented
in vectors as `NA`.

When doing operations on numbers, most functions will return `NA` if the data
you are working with include missing values. This feature
makes it harder to overlook the cases where you are dealing with missing data.
You can add the argument `na.rm = TRUE` to calculate the result as if the missing
values were \hlkwd{removed} (`rm` stands for ReMoved) first.

```\{r, purl = FALSE\}
heights <- \hlkwd{c}(2, 4, 4, NA, 6)
\hlkwd{mean}(heights)
\hlkwd{max}(heights)
\hlkwd{mean}(heights, na.rm = TRUE)
\hlkwd{max}(heights, na.rm = TRUE)
```

If your data include missing values, you may want to become familiar with the
functions `\hlkwd{is.na}()`, `\hlkwd{na.omit}()`, and `\hlkwd{complete.cases}()`. See below for
examples.


```\{r, purl = FALSE\}
\hlcom{## Extract those elements which are not missing values.}
heights[!\hlkwd{is.na}(heights)]

\hlcom{## Returns the object with incomplete cases removed.}
\hlcom{#The returned object is an atomic vector of type `"numeric"` (or #`"double"`).}
\hlkwd{na.omit}(heights)

\hlcom{## Extract those elements which are complete cases.}
\hlcom{#The returned object is an atomic vector of type `"numeric"` (or #`"double"`).}
heights[\hlkwd{complete.cases}(heights)]
```
Recall that you can use the `\hlkwd{typeof}()` function to find the type of your atomic vector.

> \hlcom{### Challenge}
>
> 1. Using this vector of heights in inches, create a new vector, `heights_no_na`, with the NAs removed.
>
>     ```r
>     heights <- \hlkwd{c}(63, 69, 60, 65, NA, 68, 61, 70, 61, 59, 64, 69, 63, 63, NA, 72, 65, 64, 70, 63, 65)
>     ```
> 2. Use the function `\hlkwd{median}()` to calculate the median of the `heights` vector.
>
> 3. Use R to figure out how many people in the set are taller than 67 inches.
>
> ```\{r, answer=TRUE\}
> heights <- \hlkwd{c}(63, 69, 60, 65, NA, 68, 61, 70, 61, 59, 64, 69, 63, 63, NA, 72, 65, 64, 70, 63, 65)
>
> \hlcom{# 1.}
> heights_no_na <- heights[!\hlkwd{is.na}(heights)]
> \hlcom{# or}
> heights_no_na <- \hlkwd{na.omit}(heights)
> \hlcom{# or}
> heights_no_na <- heights[\hlkwd{complete.cases}(heights)]
>
> \hlcom{# 2.}
> \hlkwd{median}(heights, na.rm = TRUE)
>
> \hlcom{# 3.}
> heights_above_67 <- heights_no_na[heights_no_na > 67]
> \hlkwd{length}(heights_above_67)
> ```


```\{r, echo = FALSE, purl = TRUE\}
\hlcom{## ### Challenge}
\hlcom{## 1. Using this vector of heights in inches, create a new vector with the NAs removed.}
\hlcom{##}
\hlcom{##    heights <- c(63, 69, 60, 65, NA, 68, 61, 70, 61, 59, 64, 69, 63, 63, NA, 72, 65, 64, 70, 63, 65)}
\hlcom{##}
\hlcom{## 2. Use the function `median()` to calculate the median of the `heights` vector.}
\hlcom{##}
\hlcom{## 3. Use R to figure out how many people in the set are taller than 67 inches.}
```

Now that we have learned how to write scripts, and the basics of R's data
structures, we are ready to start working with the Portal dataset we have been
using in the other lessons, and learn about data frames.


```\{r, child=\hlstr{"_page_built_on.Rmd"}\}
```
\end{alltt}


{\ttfamily\noindent\bfseries\color{errorcolor}{\#\# Error: attempt to use zero-length variable name}}\end{kframe}
\end{knitrout}

The R session information (including the OS info, R version and all
packages used):

\begin{knitrout}
\definecolor{shadecolor}{rgb}{0.969, 0.969, 0.969}\color{fgcolor}\begin{kframe}
\begin{alltt}
\hlkwd{sessionInfo}\hlstd{()}
\end{alltt}
\begin{verbatim}
## R version 4.2.1 (2022-06-23 ucrt)
## Platform: x86_64-w64-mingw32/x64 (64-bit)
## Running under: Windows 10 x64 (build 19045)
## 
## Matrix products: default
## 
## locale:
## [1] LC_COLLATE=English_New Zealand.utf8  LC_CTYPE=English_New Zealand.utf8   
## [3] LC_MONETARY=English_New Zealand.utf8 LC_NUMERIC=C                        
## [5] LC_TIME=English_New Zealand.utf8    
## 
## attached base packages:
## [1] stats     graphics  grDevices utils     datasets  methods   base     
## 
## other attached packages:
##  [1] ggrepel_0.9.2      ggthemes_4.2.4     sf_1.0-9           leaflet_2.1.1     
##  [5] fpp_0.5            tseries_0.10-53    lmtest_0.9-40      zoo_1.8-11        
##  [9] expsmooth_2.3      fma_2.4            forecast_8.20      seasonal_1.9.0    
## [13] lubridate_1.9.1    timetk_2.8.2       yardstick_1.1.0    workflowsets_1.0.0
## [17] workflows_1.1.2    tune_1.0.1         rsample_1.1.1      recipes_1.0.4     
## [21] parsnip_1.0.3      modeldata_1.1.0    infer_1.0.4        dials_1.1.0       
## [25] scales_1.2.1       broom_1.0.3        tidymodels_1.0.0   modeltime_1.2.4   
## [29] flextable_0.8.5    XKCDdata_0.1.0     forcats_1.0.0      stringr_1.5.0     
## [33] dplyr_1.1.0        purrr_1.0.1        readr_2.1.3        tidyr_1.3.0       
## [37] tibble_3.1.8       ggplot2_3.4.0      tidyverse_1.3.2    knitr_1.42        
## 
## loaded via a namespace (and not attached):
##   [1] utf8_1.2.3           tidyselect_1.2.0     htmlwidgets_1.6.1    grid_4.2.1          
##   [5] munsell_0.5.0        units_0.8-1          codetools_0.2-18     xgboost_1.7.3.1     
##   [9] future_1.31.0        withr_2.5.0          colorspace_2.1-0     highr_0.10          
##  [13] uuid_1.1-0           rstudioapi_0.14      stats4_4.2.1         wk_0.7.1            
##  [17] officer_0.5.2        TTR_0.24.3           listenv_0.9.0        labeling_0.4.2      
##  [21] rstan_2.21.8         DiceDesign_1.9       farver_2.1.1         parallelly_1.34.0   
##  [25] vctrs_0.5.2          generics_0.1.3       ipred_0.9-13         xfun_0.37           
##  [29] timechange_0.2.0     R6_2.5.1             lhs_1.1.6            cachem_1.0.6        
##  [33] assertthat_0.2.1     promises_1.2.0.1     nnet_7.3-17          googlesheets4_1.0.1 
##  [37] gtable_0.3.1         globals_0.16.2       processx_3.8.0       timeDate_4022.108   
##  [41] rlang_1.0.6          systemfonts_1.0.4    splines_4.2.1        lazyeval_0.2.2      
##  [45] gargle_1.3.0         inline_0.3.19        s2_1.1.2             yaml_2.3.7          
##  [49] modelr_0.1.10        crosstalk_1.2.0      backports_1.4.1      httpuv_1.6.8        
##  [53] quantmod_0.4.20      tools_4.2.1          lava_1.7.1           ellipsis_0.3.2      
##  [57] proxy_0.4-27         Rcpp_1.0.10          base64enc_0.1-3      classInt_0.4-8      
##  [61] ps_1.7.2             prettyunits_1.1.1    rpart_4.1.16         openssl_2.0.5       
##  [65] fracdiff_1.5-2       haven_2.5.1          fs_1.6.0             tinytex_0.44        
##  [69] furrr_0.3.1          crul_1.3             magrittr_2.0.3       data.table_1.14.6   
##  [73] reprex_2.0.2         GPfit_1.0-8          googledrive_2.0.0    x13binary_1.1.57-3  
##  [77] matrixStats_0.63.0   hms_1.1.2            mime_0.12            evaluate_0.20       
##  [81] xtable_1.8-4         readxl_1.4.1         gridExtra_2.3        compiler_4.2.1      
##  [85] KernSmooth_2.23-20   crayon_1.5.2         StanHeaders_2.21.0-7 htmltools_0.5.4     
##  [89] later_1.3.0          tzdb_0.3.0           RcppParallel_5.1.6   DBI_1.1.3           
##  [93] dbplyr_2.3.0         MASS_7.3-57          Matrix_1.5-3         cli_3.6.0           
##  [97] quadprog_1.5-8       parallel_4.2.1       gower_1.0.1          pkgconfig_2.0.3     
## [101] plotly_4.10.1        xml2_1.3.3           foreach_1.5.2        hardhat_1.2.0       
## [105] prodlim_2019.11.13   rvest_1.0.3          snakecase_0.11.0     callr_3.7.3         
## [109] digest_0.6.31        janitor_2.2.0        httpcode_0.3.0       rmarkdown_2.20      
## [113] cellranger_1.1.0     gdtools_0.3.0        curl_5.0.0           shiny_1.7.4         
## [117] urca_1.3-3           lifecycle_1.0.3      nlme_3.1-157         jsonlite_1.8.4      
## [121] viridisLite_0.4.1    askpass_1.1          fansi_1.0.4          pillar_1.8.1        
## [125] lattice_0.20-45      loo_2.5.1            fastmap_1.1.0        httr_1.4.4          
## [129] pkgbuild_1.4.0       survival_3.3-1       glue_1.6.2           xts_0.12.2          
## [133] zip_2.2.2            iterators_1.0.14     class_7.3-20         stringi_1.7.12      
## [137] prophet_1.0          gfonts_0.2.0         memoise_2.0.1        e1071_1.7-13        
## [141] future.apply_1.10.0
\end{verbatim}
\begin{alltt}
\hlkwd{Sys.time}\hlstd{()}
\end{alltt}
\begin{verbatim}
## [1] "2023-02-03 13:23:01 NZDT"
\end{verbatim}
\end{kframe}
\end{knitrout}


\end{document}
